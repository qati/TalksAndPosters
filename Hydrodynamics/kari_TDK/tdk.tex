\documentclass{beamer}
%\mode<presentation>
\usepackage[utf8]{inputenc}
\usepackage[magyar]{babel}
\usetheme{CambridgeUS}
\usecolortheme{dolphin}
\usepackage{amsmath,amssymb,amsfonts, bm}
\usepackage{mathpazo}
\usepackage{graphicx,tabularx,epsfig}
\usepackage{subcaption}
\usepackage{rotating}
\DeclareGraphicsExtensions{.pdf,.png,.jpg,.svg}


\title[Numerikus hidrodinamika]{Nagyenergiás nehézion-ütközések numerikus hidrodinamikai modellezése}
\author[Bagoly Attila]{Bagoly Attila \vspace{0.5cm}\\  Témavezető:\\ Csanád Máté}
\date[2014.12.13.]{Tudományos Diákköri Konferencia\\ 2014.12.13.}
\institute[ELTE]{ELTE TTK Atomfizikai tanszék}
\begin{document}

\begin{frame}
  \titlepage
\end{frame}

\begin{frame}
\frametitle{Motiváció}
Kvark-gluon plazma folyadékszerű viselkedésének következtében:
\vspace{20pt}
\begin{itemize}
  \setlength{\itemsep}{16pt}

\item<1-> hogyan hatnak különböző effektusok az aszimmetriák időfejlődésére
\item<1-> olyan effektusok amelyek nem kezelhetőek analitikusan
\item<2-> Numerikus hidrodinamika: realisztikus modell QGP-re (kf. pl. Monte-Carlo szimulációból): ekkor minden effektus hatása keveredik
\item<3-> Kezdőfeltétel: legyen közel létező analitikus megoldásokhoz, de realisztikusabb modellt adjon
\end{itemize}
\end{frame}

\begin{frame}
\frametitle{Tartalom}
\tableofcontents
\end{frame}

\section{Bevezető}
\begin{frame}
\frametitle{Bevezető}
\begin{minipage}{0.37\textwidth}
		\includegraphics<2->[scale=0.20]{pic/osrobbanas}
	
		\includegraphics<5->[scale=0.24]{pic/p2}
		\begin{itemize}
		\item[]
		\end{itemize}
  \end{minipage}
  \begin{minipage}{0.62\textwidth}
  \begin{itemize}
\item<1-> Nehézion ütközések: nagy energiasűrűség $\rightarrow$  kvark szabadsági fokok
\item<2-> Ősrobbanás: univerzum kvarkok és gluonok ``őslevese''
\item<2-> Új ismeretek: kvark-gluon plazma vizsgálata
\item<3-> Kísérleti tapasztalat (2005): tökéletes folyadék
\item<4-> Nagy hatáskeresztmetszetek, rövid szabad úthossz, gyors termalizáció: statisztikus fizika
\item<5-> Kezdeti eloszlás: aszimmetriák a kezdeti eloszlásban
\item<6-> Aszimmetriák: kifagynak a részecskék eloszlásában
\end{itemize}

\end{minipage}
\end{frame}

\section{Hidrodinamika egyenletei}
\begin{frame}
\frametitle{Hidrodinamika egyenletei}
\begin{itemize}
\item<1-> Nemrelativisztikus hidrodinamika:
\begin{itemize}
\item<2-> Anyagmegmaradás:\quad \begin{large}$\dfrac{\partial \rho}{\partial t} + \bm{\nabla}\rho\bm{v}=0$\end{large}
\item<3-> Impulzusmegmaradás:\quad \begin{center}\begin{large}$\rho\Big(\dfrac{\partial \bm{v}}{\partial t}+(\bm{v\nabla})\bm{v}\Big)=-\bm{\nabla}p +\mu\Delta\bm{v}+\Big(\zeta+\dfrac{\mu}{3}\Big)\bm{\nabla}(\bm{\nabla v})+\bm{f}$\end{large}\end{center}
\item<4-> Energiamegmaradás: \quad\begin{large}$\dfrac{\partial \varepsilon}{\partial t}+\bm{\nabla}\varepsilon\bm{v}=-p\bm{\nabla v}+\bm{\nabla(\sigma v)}$\end{large}
\item<4-> $\rho$ anyagsűrűség, $\bm{v}$ sebességmező, $\varepsilon$ energiasűrűség, $p$ nyomáseloszlás
\end{itemize}


\item<5-> Állapotegyenlet:\quad \begin{large}$\varepsilon=\kappa(T)p\quad$\end{large} ($\kappa=1/c_s^2$, $\kappa=3/2$ id. gáz)

\item<6-> Relativisztikus hidrodinamika:
\begin{Large}
\begin{center}$T^{\mu\nu}=\big(\varepsilon+p\big)u^\mu u^\nu-pg^{\mu\nu},\quad \partial_\mu T^{\mu\nu}=0$\end{center}
\end{Large}
\begin{itemize}
\item<6-> $T^{\mu\nu}$ energia-impulzus tenzor, $u^\mu$ négyes-sebesség, $g^{\mu\nu}$ metrikus tenzor
\end{itemize}
\item<7-> Advekciós forma: $\partial_t Q_i+\partial_x F_i(Q)+\partial_y G_i(Q)+\partial_z K_i(Q)=0$

\end{itemize}
\end{frame}


%\begin{frame}
%\frametitle{Multipole solution}
%  \begin{minipage}{0.39\textwidth}
%    \includegraphics[scale=0.16]{pic/a1}

%    \includegraphics[scale=0.2]{pic/a2}
%  \end{minipage}
%  \begin{minipage}{0.60\textwidth}
%  \begin{itemize}
%\item New exact solution of relativistic hydrodynamics by Máté Csanád and András Szabó, published Phys.Rev. C90 (2014) 054911
%\item The solution in cylindrical coordinates:
%$u^\mu=\frac{x^\mu}{\tau}$, $n=n_f\big(\frac{\tau_f}{\tau}\big)^3\nu(s)$, $p=p_f\big(\frac{\tau_f}{\tau}\big)^{3+3/\kappa}$
%\item Where $\tau$ is the coordinate-proper time, $\tau_f$ is the freeze-out proper time
%\item The s scale variable with any asymmetries 
%$s=\frac{r^N}{R^N}\Big(1+\epsilon_{N}\cos N\phi\Big)+\frac{z^N}{R^N}$
%\item That's a solution if $R=u_tt$
%\end{itemize}
%  \end{minipage}
%\end{frame}


\section{Numerikus módszer}
\begin{frame}
\frametitle{Numerikus módszer}
\begin{itemize}
  \setlength{\itemsep}{2pt}

\item<1-> Midrapiditás: eloszlások maximuma, sőt ``plató'' $\rightarrow$ $2+1$ dimenzió
\item<2-> Numerikus megoldás: diszkretizáció
\item<3-> Véges térfogat módszer: mennyiségek átlaga rácspont körül
\item<4-> Probléma: fluxusok értéke a rácspontok között, egzaktul nem lehet
\item<5-> Instabilitás: $Q_i+Ae^{-i\bm{k}\Delta\bm{x}}$ $\rightarrow$ CFL feltétel (pl. $C=u\Delta t/ \Delta x < 1$)
\item<6-> 2 térdimenziót bonyolult kezelni $\rightarrow$ operátor szétválasztás
\item<7-> Viszkozitás: ideális lépés + csak viszkózus fluxussal lépés
\end{itemize}
\begin{center}
\includegraphics<4->[scale=0.19]{pic/flux}
\end{center}
\end{frame}

\begin{frame}
\frametitle{MUSTA módszer}
\begin{itemize}
\item<1-> Az $l$ előrejelzett értékek: $Q^{(l)}_{i/(i+1)}$, $F^{(l)}_{i/(i+1)}\equiv F\big(Q^{(l)}_{i/(i+1)}\big)$
\item<2-> Kezdetben: $Q^{(0)}_i\equiv Q^n_i$,  $Q^{(0)}_{i+1}\equiv Q^{n}_{i+1}$
\item<3-> Köztes érték és fluxus:
\begin{equation*}
Q^{(l)}_{i+\frac{1}{2}}=\frac{1}{2}\Big[Q^{(l)}_i+Q^{(l)}_{i+1}\Big]-\frac{1}{2}\frac{\Delta t}{\Delta x}\Big[F^{(l)}_{i+1}-F^{(l)}_i\Big], \quad F_M^{(l)} \equiv F\big(Q^{(l)}_{i+\frac{1}{2}}\big)
\end{equation*}

\item<3-> Korrigált cellaközi fluxus:
\begin{equation*}
F^{(l)}_{i+\frac{1}{2}}=\frac{1}{4}\Big[F^{(l)}_{i+1}+2F^{(l)}_M+F^{(l)}_{i}-\frac{\Delta x}{\Delta t}\Big(Q^{(l)}_{i+1}-Q^{(l)}_i\Big)\Big]
\end{equation*}
\item<4-> Következő előrejelzés a korrigált fluxusok meghatározásához:
\begin{equation*}
Q_i^{(l+1)}=Q^{(l)}_i-\frac{\Delta t}{\Delta x}\Big[F^{(l)}_{i+\frac{1}{2}}-F^{(l)}_i\Big]
\end{equation*}
\begin{equation*}
Q_{i+1}^{(l+1)}=Q^{(l)}_{i+1}-\frac{\Delta t}{\Delta x}\Big[F^{(l)}_{i+1}-F^{(l)}_{i+\frac{1}{2}}\Big]
\end{equation*}
\item<4-> A módszer publikálva: E. F. Toro et al, 2006, J. Comp. Phys
\end{itemize}
\end{frame}

\section{Kód tesztelése}
\begin{frame}
\frametitle{Kód tesztelése}
\begin{itemize}
  \setlength{\itemsep}{5pt}
\item<1-> A kódot meglévő egzakt megoldásokkal teszteltük (Csörgő et al, PhysRevC67):
$s=\frac{x^2}{X^2(t)}+\frac{y^2}{Y^2(t)}$, $\rho = \rho_0\frac{V_0}{V}e^{-s}$,  $p=p_0\big(\frac{V_0}{V}\big)^{1+\frac{1}{\kappa}}$, $\bm{v}(t, \bm{r})=\big(\frac{\dot{X}}{X}x, \frac{\dot{Y}}{Y}y\big)$, $\ddot{X}X=\ddot{Y}Y=\frac{T_i}{m}\big(\frac{V_0}{V}\big)^{\frac{1}{\kappa}}$, $V=X(t)Y(t)$
\item<2-> Relatív hiba a numerikus és analitikus megoldás közt
\end{itemize}
\begin{center}
\visible<2->{$X=Y$ \qquad\qquad\qquad\qquad\qquad\qquad\qquad $X\neq Y$}
\begin{figure}[H]
	\centering
    \begin{subfigure}[b]{0.49\textwidth}
    		\includegraphics<2->[width=\textwidth]{pic/sym}
	\end{subfigure}
	\begin{subfigure}[b]{0.49\textwidth}
        	\includegraphics<2->[width=\textwidth]{pic/asym}
	\end{subfigure}
\end{figure}
\end{center}
\end{frame}


\section{}
\begin{frame}
\frametitle{Kezdőfeltétel}
\begin{itemize}
\item<1-> Skálaváltozó: 
\begin{equation*}
s=\frac{r^2}{R^2}\Big(1+\epsilon_2\cos(2\phi)+\epsilon_3\cos(3\phi)+\epsilon_4\cos(4\phi)\Big)
\end{equation*}
\begin{center}
\includegraphics[scale=0.15]{pic/a1}
\end{center}
\item<2-> Sebesség: nemrelativisztikus estben $0$, relativisztikus esetben Hubble-sebességmező
\item<2-> Anyagűrűség/számsűrűség és nyomás $\propto \exp{(-s)}$
\item<3-> Konstans nyomással analitikus megoldás: Csanád és Szabó, Phys.Rev. C90 (2014) 054911
\item<3-> Nyomásgradiens vizsgálata: $p \propto \exp{(-\rm{pc}\cdot s)}$
\end{itemize}
\end{frame}

\begin{frame}
\frametitle{Aszimmetriák jellemzése}
\begin{itemize}
  \setlength{\itemsep}{12pt}

\item<1-> Aszimmetriát jellemző paraméterre: $\varepsilon_n=\langle \cos(n\phi)\rangle_{\rho/\bm{v}/p}$
\item<2-> $\varepsilon_n$ (most bevezetett)  $\neq$ $\epsilon_m$  (kezdőfeltétel skálaváltozójában)
\item<3-> Kezdetben a $\varepsilon_n$ és $\epsilon_m$ közti kapcsolatot becsülhetjük Taylor-sorfejtéssel:
\vspace{10pt}
\begin{itemize}
 \setlength{\itemsep}{8pt}
\item<3-> Legyen: $\chi=2+\sum_{n=2}^4\epsilon_n^2$
\item<3-> Megjelenik: $\varepsilon_1=(\epsilon_2+\epsilon_4)\epsilon_3/\chi$
\item<3-> $\epsilon_4$ befolyásolja: $\varepsilon_2=\epsilon_2(\epsilon_4-1)/\chi$
\item<3-> $\varepsilon_3=-\epsilon_3/\chi$
\item<3-> $\epsilon_2$ is létrehoz: $\varepsilon_4=(-\epsilon_4+\frac{1}{2}\epsilon_2^2)/\chi$
\end{itemize}
\end{itemize}
\end{frame}

\section{Nemrelativisztikus eredmények}

\begin{frame}
\frametitle{Viszkozitás hatása}
\begin{center}
\begin{itemize}
  \setlength{\itemsep}{2pt}

\item<1-> Energiasűrűségben és  anyagsűrűségben: lassít
\begin{itemize}
\item<1-> Viszkozitás: lassítja az áramlást
\end{itemize}
\item<2-> Sebességeloszlásban: gyorsít
\begin{itemize}
\item<2-> Nagyobb,kisebb aszimmetriájú részek más erőt éreznek: különbségek gyorsan eltűnnek
\end{itemize}
\item Ábra: \large{\textcolor{red}{$\varepsilon_1$ piros}, \textcolor{green}{$\varepsilon_2$ zöld}, \textcolor{blue}{$\varepsilon_3$ kék},  \textcolor{magenta}{$\varepsilon_4$ magenta}}
\end{itemize}
\begin{figure}[H]
	\centering
    \begin{subfigure}[b]{0.49\textwidth}
    		\includegraphics[width=\textwidth]{pic/res/nonrel/eps_visc_r}
	\end{subfigure}
	\begin{subfigure}[b]{0.49\textwidth}
        	\includegraphics[width=\textwidth]{pic/res/nonrel/eps_visc_v}
	\end{subfigure}
\end{figure}
\end{center}
\end{frame}

\begin{frame}
\frametitle{Viszkozitás hatása: energiasűrűség időfejlődése}
\begin{minipage}{0.04\textwidth}
\rotatebox[origin=c]{90}{$\qquad$\scalebox{0.85}{$\mu=0\;\textnormal{MeVfm}/c$}}
\rotatebox[origin=c]{90}{\scalebox{0.85}{$\mu=10\;\textnormal{MeVfm}/c$}}
\end{minipage}
\begin{minipage}{0.95\textwidth}
\begin{center}
    \includegraphics[scale=0.19]{pic/res/nonrel/anim/ev0}

    \includegraphics[scale=0.19]{pic/res/nonrel/anim/ev10}
\end{center}
\end{minipage}
\end{frame}
\begin{frame}
\frametitle{Viszkozitás hatása:  sebességeloszlás időfejlődése}
\begin{minipage}{0.04\textwidth}
\rotatebox[origin=c]{90}{$\qquad$\scalebox{0.85}{$\mu=0\;\textnormal{MeVfm}/c$}}
\rotatebox[origin=c]{90}{\scalebox{0.85}{$\mu=10\;\textnormal{MeVfm}/c$}}
\end{minipage}
\begin{minipage}{0.95\textwidth}
\begin{center}
    \includegraphics[scale=0.15]{pic/res/nonrel/anim/vv0}

    \includegraphics[scale=0.15]{pic/res/nonrel/anim/vv10}
\end{center}
\end{minipage}
\end{frame}

\begin{frame}
\frametitle{Hangsebesség hatása}
\begin{center}
\begin{itemize}
\setlength{\itemsep}{12pt}
\item<1-> Minden eloszlásban: aszimmetriák eltűnése lassul
\vspace{8pt}
\begin{itemize}
\item<1-> Nyomáshullámok sebessége csökken $\rightarrow$ kiegyenlítődés tovább tart
\end{itemize}
\item Hangsebességek: $c_s^2=1\; \rm{vagy}\;0,4\; \rm{vagy}\; \rm{0,33}\; vagy\; \rm{0,25}$
\end{itemize}
\begin{figure}[H]
	\centering
    \begin{subfigure}[b]{0.49\textwidth}
    		\includegraphics[width=\textwidth]{pic/res/nonrel/eps_cs2_r}
	\end{subfigure}
	\begin{subfigure}[b]{0.49\textwidth}
        	\includegraphics[width=\textwidth]{pic/res/nonrel/eps_cs2_v}
	\end{subfigure}
\end{figure}
\end{center}
\end{frame}

\begin{frame}
\frametitle{Nyomásgradiens hatása}
\begin{center}
\begin{itemize}
\setlength{\itemsep}{12pt}
\item<1-> Minden eloszlásban: aszimmetriák gyorsabban eltűnnek
\begin{itemize}
\vspace{8pt}
\item<1-> Nagyobb gradiens: gyorsabb áramlás
\end{itemize}
\item<1-> Anyagsűrűség $\propto \exp{(-s)}$
\item<1-> Energiasűrűség $\propto \exp{(-\rm{ec}\cdot s)}$
\end{itemize}
\begin{figure}[H]
	\centering
    \begin{subfigure}[b]{0.49\textwidth}
    		\includegraphics[width=\textwidth]{pic/res/nonrel/eps_ec_r}
	\end{subfigure}
	\begin{subfigure}[b]{0.49\textwidth}
        	\includegraphics[width=\textwidth]{pic/res/nonrel/eps_ec_v}
	\end{subfigure}
\end{figure}
\end{center}
\end{frame}

\section{Relativisztikus eredmények}
\begin{frame}
\frametitle{Hangsebesség hatása}
\begin{center}
\begin{itemize}
\setlength{\itemsep}{12pt}
\item<1-> Minden eloszlásban: aszimmetriák eltűnése lassul
\vspace{8pt}
\begin{itemize}
\item<1-> Nyomáshullámok sebessége csökken $\rightarrow$ kiegyenlítődés tovább tart
\end{itemize}
\item<2-> Kifagyás máskor történik!
\end{itemize}
\begin{figure}[H]
	\centering
    \begin{subfigure}[b]{0.49\textwidth}
    		\includegraphics[width=\textwidth]{pic/res/rel/eps_kappa_n}
	\end{subfigure}
	\begin{subfigure}[b]{0.49\textwidth}
        	\includegraphics[width=\textwidth]{pic/res/rel/eps_kappa_v}
	\end{subfigure}
\end{figure}
\end{center}
\end{frame}

\begin{frame}
\frametitle{Nyomásgradiens hatása}
\begin{center}
\begin{itemize}
\setlength{\itemsep}{12pt}
\item<1-> Minden eloszlásban: aszimmetriák gyorsabban eltűnnek
\begin{itemize}
\vspace{8pt}
\item<1-> Nagyobb gradiens: gyorsabb áramlás
\end{itemize}
\item<1-> Számsűrűség $\propto \exp{(-s)}$
\item<1-> Nyomás $\propto \exp{(-\rm{pc}\cdot s)}$
\end{itemize}
\begin{figure}[H]
	\centering
    \begin{subfigure}[b]{0.49\textwidth}
    		\includegraphics[width=\textwidth]{pic/res/rel/eps_pc_n}
	\end{subfigure}
	\begin{subfigure}[b]{0.49\textwidth}
        	\includegraphics[width=\textwidth]{pic/res/rel/eps_pc_v}
	\end{subfigure}
\end{figure}
\end{center}
\end{frame}

\section{Kifagyás}
\begin{frame}
\frametitle{Kifagyás}
\begin{itemize}
\item<1-> Maxwell-Jüttner típusú forrásfüggvény: 
$S(x, p)d^4x=\mathcal{N}n(x)\exp{\bigg(-\frac{p_\mu u^\mu}{T(x)}\bigg)}H(\tau)p_\mu d^3\frac{u_\mu d^3x}{u^0} d\tau$
\item<2-> Mérhető mennyiségek:
$v_n(p_t)=\langle\cos(n\varphi)\rangle_{N}=\frac{1}{N(p_t)}\int_0^{2\pi} N(p_t, \varphi)\cos(n\varphi)d\varphi$
\end{itemize}
\begin{minipage}{0.49\textwidth}
\begin{center}
\includegraphics<2->[scale=0.23]{pic/res/rel/vn_kappa}
\end{center}
\end{minipage}
\begin{minipage}{0.5\textwidth}
\begin{itemize}
\item<2-> Impulzustérbeli aszimmetriák: erősen függés a hangsebességtől
\item<2-> Hangsebességre érzékeny: kifagyás ideje
\end{itemize}
\end{minipage}
\end{frame}

\section{Összegzés}
\begin{frame}
\frametitle{Összegzés}
\begin{itemize}
\setlength{\itemsep}{16pt}
\item<1-> Motiváció: egyszerű effektusok, hogyan befolyásolják az aszimmetriák időfejlődését
\item<2-> Analitikus tárgyalásra kevés az esély, ezért numerikus módszert alkalmaztunk
\item<3-> Kezdőfeltétel hasonló a már létező analitikus megoldásokhoz, de realisztikusabb
\item<4-> A viszkozitás lassabbá teszi az anyag- és energiasűrűségben számolt aszimmetriák időfejlődését, sebességeloszlásban gyorsabbá
\item<5-> Hangsebesség csökkentése lassítja az aszimmetriák időfejlődését, kifagyás később következik be
\end{itemize}
\end{frame}

\section{}
\begin{frame}
\begin{center}
\textbf{\huge{Köszönöm a figyelmet!}}
\end{center}
\end{frame}

\begin{frame}[noframenumbering]
\frametitle{Viszkozitás hatása: $\varepsilon_n$ anyag- és energiasűrűségben}
\begin{center}
\begin{figure}[H]
	\centering
    \begin{subfigure}[b]{0.49\textwidth}
    		\includegraphics[width=\textwidth]{pic/res/nonrel/eps_visc_r}
	\end{subfigure}
	\begin{subfigure}[b]{0.49\textwidth}
        	\includegraphics[width=\textwidth]{pic/res/nonrel/eps_visc_p}
	\end{subfigure}
\end{figure}
\end{center}
\end{frame}

\begin{frame}[noframenumbering]
\frametitle{Hangsebesség hatása: $\varepsilon_n$ anyag- és energiasűrűségben}
\begin{center}
\begin{figure}[H]
	\centering
    \begin{subfigure}[b]{0.49\textwidth}
    		\includegraphics[width=\textwidth]{pic/res/nonrel/eps_cs2_r}
	\end{subfigure}
	\begin{subfigure}[b]{0.49\textwidth}
        	\includegraphics[width=\textwidth]{pic/res/nonrel/eps_cs2_p}
	\end{subfigure}
\end{figure}
\end{center}
\end{frame}

\begin{frame}[noframenumbering]
\frametitle{Nyomásgradiens hatása: $\varepsilon_n$ anyag- és energiasűrűségben}
\begin{center}
\begin{figure}[H]
	\centering
    \begin{subfigure}[b]{0.49\textwidth}
    		\includegraphics[width=\textwidth]{pic/res/nonrel/eps_ec_r}
	\end{subfigure}
	\begin{subfigure}[b]{0.49\textwidth}
        	\includegraphics[width=\textwidth]{pic/res/nonrel/eps_ec_p}
	\end{subfigure}
\end{figure}
\end{center}
\end{frame}

\begin{frame}[noframenumbering]
\frametitle{Operátorok felbontása}
\begin{large}
\begin{equation*}
\partial_t u = Au+Bu
\end{equation*}

\begin{equation*}
u(t+\Delta t)=e^{\Delta t(A+B)}u(t)
\end{equation*}


\begin{equation*}
u_{\rm{Lie}}(t+\Delta t)=e^{\Delta tA}e^{\Delta tB}u(t)
\end{equation*}

\begin{equation*}
u_{\rm{Strang}}(t+\Delta t)=e^{\frac{1}{2}\Delta tA}e^{\Delta tB}e^{\frac{1}{2}\Delta tA}e^{\Delta tB}u(t)
\end{equation*}
\end{large}
\end{frame}


\end{document}

